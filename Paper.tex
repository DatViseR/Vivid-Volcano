% Options for packages loaded elsewhere
\PassOptionsToPackage{unicode}{hyperref}
\PassOptionsToPackage{hyphens}{url}
%
\documentclass[
]{article}
\usepackage{amsmath,amssymb}
\usepackage{iftex}
\ifPDFTeX
  \usepackage[T1]{fontenc}
  \usepackage[utf8]{inputenc}
  \usepackage{textcomp} % provide euro and other symbols
\else % if luatex or xetex
  \usepackage{unicode-math} % this also loads fontspec
  \defaultfontfeatures{Scale=MatchLowercase}
  \defaultfontfeatures[\rmfamily]{Ligatures=TeX,Scale=1}
\fi
\usepackage{lmodern}
\ifPDFTeX\else
  % xetex/luatex font selection
\fi
% Use upquote if available, for straight quotes in verbatim environments
\IfFileExists{upquote.sty}{\usepackage{upquote}}{}
\IfFileExists{microtype.sty}{% use microtype if available
  \usepackage[]{microtype}
  \UseMicrotypeSet[protrusion]{basicmath} % disable protrusion for tt fonts
}{}
\makeatletter
\@ifundefined{KOMAClassName}{% if non-KOMA class
  \IfFileExists{parskip.sty}{%
    \usepackage{parskip}
  }{% else
    \setlength{\parindent}{0pt}
    \setlength{\parskip}{6pt plus 2pt minus 1pt}}
}{% if KOMA class
  \KOMAoptions{parskip=half}}
\makeatother
\usepackage{xcolor}
\usepackage[margin=1in]{geometry}
\usepackage{graphicx}
\makeatletter
\def\maxwidth{\ifdim\Gin@nat@width>\linewidth\linewidth\else\Gin@nat@width\fi}
\def\maxheight{\ifdim\Gin@nat@height>\textheight\textheight\else\Gin@nat@height\fi}
\makeatother
% Scale images if necessary, so that they will not overflow the page
% margins by default, and it is still possible to overwrite the defaults
% using explicit options in \includegraphics[width, height, ...]{}
\setkeys{Gin}{width=\maxwidth,height=\maxheight,keepaspectratio}
% Set default figure placement to htbp
\makeatletter
\def\fps@figure{htbp}
\makeatother
\setlength{\emergencystretch}{3em} % prevent overfull lines
\providecommand{\tightlist}{%
  \setlength{\itemsep}{0pt}\setlength{\parskip}{0pt}}
\setcounter{secnumdepth}{-\maxdimen} % remove section numbering
% definitions for citeproc citations
\NewDocumentCommand\citeproctext{}{}
\NewDocumentCommand\citeproc{mm}{%
  \begingroup\def\citeproctext{#2}\cite{#1}\endgroup}
\makeatletter
 % allow citations to break across lines
 \let\@cite@ofmt\@firstofone
 % avoid brackets around text for \cite:
 \def\@biblabel#1{}
 \def\@cite#1#2{{#1\if@tempswa , #2\fi}}
\makeatother
\newlength{\cslhangindent}
\setlength{\cslhangindent}{1.5em}
\newlength{\csllabelwidth}
\setlength{\csllabelwidth}{3em}
\newenvironment{CSLReferences}[2] % #1 hanging-indent, #2 entry-spacing
 {\begin{list}{}{%
  \setlength{\itemindent}{0pt}
  \setlength{\leftmargin}{0pt}
  \setlength{\parsep}{0pt}
  % turn on hanging indent if param 1 is 1
  \ifodd #1
   \setlength{\leftmargin}{\cslhangindent}
   \setlength{\itemindent}{-1\cslhangindent}
  \fi
  % set entry spacing
  \setlength{\itemsep}{#2\baselineskip}}}
 {\end{list}}
\usepackage{calc}
\newcommand{\CSLBlock}[1]{\hfill\break\parbox[t]{\linewidth}{\strut\ignorespaces#1\strut}}
\newcommand{\CSLLeftMargin}[1]{\parbox[t]{\csllabelwidth}{\strut#1\strut}}
\newcommand{\CSLRightInline}[1]{\parbox[t]{\linewidth - \csllabelwidth}{\strut#1\strut}}
\newcommand{\CSLIndent}[1]{\hspace{\cslhangindent}#1}
\ifLuaTeX
  \usepackage{selnolig}  % disable illegal ligatures
\fi
\usepackage{bookmark}
\IfFileExists{xurl.sty}{\usepackage{xurl}}{} % add URL line breaks if available
\urlstyle{same}
\hypersetup{
  pdftitle={Vivid Volcano: Empowering Non-Bioinformaticians to Analyze Pre-Processed Omics Data},
  hidelinks,
  pdfcreator={LaTeX via pandoc}}

\title{Vivid Volcano: Empowering Non-Bioinformaticians to Analyze
Pre-Processed Omics Data}
\author{}
\date{\vspace{-2.5em}20 March 2025}

\begin{document}
\maketitle

\section{Summary}\label{summary}

The forces on stars, galaxies, and dark matter under external
gravitational fields lead to the dynamical evolution of structures in
the universe. The orbits of these bodies are therefore key to
understanding the formation, history, and future state of galaxies. The
field of ``galactic dynamics,'' which aims to model the gravitating
components of galaxies to study their structure and evolution, is now
well-established, commonly taught, and frequently used in astronomy.
Aside from toy problems and demonstrations, the majority of problems
require efficient numerical tools, many of which require the same base
code (e.g., for performing numerical orbit integration).

\section{Statement of need}\label{statement-of-need}

\texttt{Gala} is an Astropy-affiliated Python package for galactic
dynamics. Python enables wrapping low-level languages (e.g., C) for
speed without losing flexibility or ease-of-use in the user-interface.
The API for \texttt{Gala} was designed to provide a class-based and
user-friendly interface to fast (C or Cython-optimized) implementations
of common operations such as gravitational potential and force
evaluation, orbit integration, dynamical transformations, and chaos
indicators for nonlinear dynamics. \texttt{Gala} also relies heavily on
and interfaces well with the implementations of physical units and
astronomical coordinate systems in the \texttt{Astropy} package (Astropy
Collaboration 2013) (\texttt{astropy.units} and
\texttt{astropy.coordinates}).

\texttt{Gala} was designed to be used by both astronomical researchers
and by students in courses on gravitational dynamics or astronomy. It
has already been used in a number of scientific publications (Pearson,
Price-Whelan, and Johnston 2017) and has also been used in graduate
courses on Galactic dynamics to, e.g., provide interactive
visualizations of textbook material (Binney and Tremaine 2008). The
combination of speed, design, and support for Astropy functionality in
\texttt{Gala} will enable exciting scientific explorations of
forthcoming data releases from the \emph{Gaia} mission (Gaia
Collaboration 2016) by students and experts alike.

\section{Mathematics}\label{mathematics}

Single dollars (\$) are required for inline mathematics
e.g.~\(f(x) = e^{\pi/x}\)

Double dollars make self-standing equations:

\[\Theta(x) = \left\{\begin{array}{l}
0\textrm{ if } x < 0\cr
1\textrm{ else}
\end{array}\right.\]

You can also use plain \LaTeX for equations
\begin{equation}\label{eq:fourier}
\hat f(\omega) = \int_{-\infty}^{\infty} f(x) e^{i\omega x} dx
\end{equation} and refer to \autoref{eq:fourier} from text.

\section{Citations}\label{citations}

Citations to entries in paper.bib should be in
\href{http://rmarkdown.rstudio.com/authoring_bibliographies_and_citations.html}{rMarkdown}
format.

If you want to cite a software repository URL (e.g.~something on GitHub
without a preferred citation) then you can do it with the example BibTeX
entry below for Smith, Thaney, and Hahnel (2020).

For a quick reference, the following citation commands can be used: -
\texttt{@author:2001} -\textgreater{} ``Author et al.~(2001)'' -
\texttt{{[}@author:2001{]}} -\textgreater{} ``(Author et al., 2001)'' -
\texttt{{[}@author1:2001;\ @author2:2001{]}} -\textgreater{} ``(Author1
et al., 2001; Author2 et al., 2002)''

\section{Figures}\label{figures}

Figures can be included like this: \includegraphics{figure.png} and
referenced from text using \autoref{fig:example}.

Figure sizes can be customized by adding an optional second parameter:
\includegraphics[width=0.2\textwidth,height=\textheight]{figure.png}

\section{Acknowledgements}\label{acknowledgements}

We acknowledge contributions from Brigitta Sipocz, Syrtis Major, and
Semyeong Oh, and support from Kathryn Johnston during the genesis of
this project.

\section*{References}\label{references}
\addcontentsline{toc}{section}{References}

\phantomsection\label{refs}
\begin{CSLReferences}{1}{0}
\bibitem[\citeproctext]{ref-astropy}
Astropy Collaboration. 2013. {``{Astropy: A community Python package for
astronomy}.''} \emph{Astronomy and Astrophysics} 558 (October).
\url{https://doi.org/10.1051/0004-6361/201322068}.

\bibitem[\citeproctext]{ref-Binney:2008}
Binney, J., and S. Tremaine. 2008. \emph{{Galactic Dynamics: Second
Edition}}. Princeton University Press.
\url{http://adsabs.harvard.edu/abs/2008gady.book.....B}.

\bibitem[\citeproctext]{ref-gaia}
Gaia Collaboration. 2016. {``{The Gaia mission}.''} \emph{Astronomy and
Astrophysics} 595 (November).
\url{https://doi.org/10.1051/0004-6361/201629272}.

\bibitem[\citeproctext]{ref-Pearson:2017}
Pearson, S., A. M. Price-Whelan, and K. V. Johnston. 2017. {``{Gaps in
Globular Cluster Streams: Pal 5 and the Galactic Bar}.''} \emph{ArXiv
e-Prints}, March.
\url{http://adsabs.harvard.edu/abs/2017arXiv170304627P}.

\bibitem[\citeproctext]{ref-fidgit}
Smith, A. M., K. Thaney, and M. Hahnel. 2020. {``Fidgit: An Ungodly
Union of GitHub and Figshare.''} \emph{GitHub Repository}. GitHub.
\url{https://github.com/arfon/fidgit}.

\end{CSLReferences}

\end{document}
